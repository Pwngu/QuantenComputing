\pagenumbering{Roman}

\begin{thebibliography}{9}

\bibitem{quanteninformatik}
    Gilbert Brands,
    \emph{Einführung in die Quanteninformatik: Quantenkryptografie, Teleportation und Quantencomputing},
    2011,
    ISBN 9783642206474.

\bibitem{cryptintroduction}
    Johannes A. Buchmann,
    \emph{Introduction to Cryptography},
    2001, 4. edition

\bibitem{experiement_qsa}
    Anton Zeilinger,
    \emph{Bank Transfer via Quantum Cryptography Based on Entangled Photons},
    2004,
    online: \url{https://web.archive.org/web/20150211032846/http://www.secoqc.net/downloads/pressrelease/Banktransfer_english.pdf};
    zuletzt abgerufen 28. März 2016.

\bibitem{symkeyresistance}
    U.S. Department of Commerce,
    \emph{Quantum Resistant Public Key Cryptography: A Survey},
    2013,
    online: \url{http://www.nist.gov/manuscript-publication-search.cfm?pub_id=901595};
    zuletzt abgerufen 31. März 2016.

\bibitem{quantenschluesselaustausch}
    Artikel mit Interaktiven Experimenten zu \emph{Quanten Schlüsselaustausch},
    \url{http://www.didaktik.physik.uni-erlangen.de/quantumlab/};
    zuletzt abgerufen 28. März 2016.

\bibitem{numberfieldsieve}
    Eric Weisstein,
    \emph{Number Field Sieve} (dt. Zahlenkörpersieb),
    2016,
    online: \url{http://mathworld.wolfram.com/NumberFieldSieve.html};
    zuletzt abgerufen 31. März 2016.


\bibitem{qcl}
    \emph{QCL - A Programming Language for Quantum Computers},
    online: \url{http://tph.tuwien.ac.at/~oemer/qcl.html};
    zuletzt abgerufen 31. März 2016.

\bibitem{lanq}
    \emph{LanQ – a quantum imperative programming language},
    online: \url{http://lanq.sourceforge.net}
    zuletzt abgerufen 31. März 2016.


\bibitem{quantencomputervideo}
    You-Tube Video über Basics von Quantencomputern,
    \url{https://www.youtube.com/watch?v=JhHMJCUmq28};
    zuletzt abgerufen 28. März 2016.


\bibitem{quantenprogwiki}
    Wikipediaartiekl zu \emph{Quantenprogrammierung},
    \url{https://en.wikipedia.org/wiki/Quantum_programming#cite_note-2},
    zuletzt abgerufen 31. März 2016.

\bibitem{onetimepadwiki}
    Wikipediaartikel zu \emph{One-Time-Pad},
    \url{https://de.wikipedia.org/wiki/One-Time-Pad};
    zuletzt abgerufen 28. März 2016.

\bibitem{shorwiki}
    Wikipediartikel zu \emph{Shor's Algorithmus}
    \url{https://de.wikipedia.org/wiki/Shor-Algorithmus};
    zuletzt abgerufen 31. März 2016.

\bibitem{groverwiki}
    Wikipediaartikel zu \emph{Grover's Algorithmus}
    \url{https://de.wikipedia.org/wiki/Grover-Algorithmus};
    zuletzt abgerufen 31. März 2016.

\bibitem{postquantumwiki}
    Wikipediaartikel zu \emph{Post-Quanten-Kryptographie},
    \url{https://de.wikipedia.org/wiki/Post-Quanten-Kryptographie};
    zuletzt abgerufen 28. März 2016.

\bibitem{quantenalgorithmgwiki}
    Wikipediaartikel zu \emph{Quantenalgorithmen},
    \url{https://en.wikipedia.org/wiki/Quantum_algorithm};
    zuletzt abgerufen 28. März 2016.

\bibitem{buch_quantumcomputing}
	Buch \emph{Quantum Computing Verstehen}
	Matthias Homeister
	2. Auflage, 2008
	ISBN 978-3-8348-0436-5
	online: \url{http://catalogplus.tuwien.ac.at/primo_library/libweb/action/display.do?tabs=detailsTab&doc=UTW_aleph_acc000449741}
	zuletzt aufgerufen: 19.11.2015

\bibitem{dwave1}
	online: \url{https://www.wired.de/collection/latest/der-quantencomputer-d-wave-scheint-zu-funktionieren}
	zuletzt aufgerufen: 29.03.2016
	
\bibitem{dwave2}
	online: \url{http://venturebeat.com/2015/12/08/google-says-its-quantum-computer-is-more-than-100-million-times-faster-than-a-regular-computer-chip/}
	zuletzt aufgerufen: 29.03.2016
	
\bibitem{dwave3}
	online: \url{http://www.spektrum.de/news/daempfer-fuer-den-d-wave-quantencomputer/1296152}
	zuletzt aufgerufen: 29.03.2016
	
\bibitem{dwave4}
	online: \url{http://motherboard.vice.com/read/google-claims-its-d-wave-quantum-computer-is-the-real-deal}
	zuletzt aufgerufen: 29.03.2016
	
\bibitem{dwave5}
	online: \url{http://www.dwavesys.com/quantum-computing}
	zuletzt aufgerufen: 29.03.2016
	
\bibitem{dwave6}
	online: \url{http://www.dwavesys.com/quantum-computing/industries}
	zuletzt aufgerufen: 29.03.2016
	
\bibitem{yt1}
	Piled Higher and Deeper (PHD Comics),
	2013,
	\emph{Quantum Computers Animated}
	online: \url{https://www.youtube.com/watch?v=T2DXrs0OpHU}
	zuletzt aufgerufen: 29.03.2016
	
\end{thebibliography}
